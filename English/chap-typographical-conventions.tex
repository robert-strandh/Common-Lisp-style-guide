\chapter{Typographical conventions}

\section{Width of a line of code}

Horizontal space is a precious resource that should not be wasted.
The width of a line should preferably not exceed 80 characters.  This
limit used to be hard, because some printers or printer drivers would
truncate longer lines.  Since it is less common to print code these
days, the limit is now soft.  The purpose of keeping lines somewhat
short is so that it is possible on a reasonable monitor to display two
documents side by side.  One document is typically a \commonlisp{}
source file, and the other document is typically the buffer containing
interactions with the \commonlisp{} system.

The systematic use of long lines makes the practice of displaying two
documents side by side impossible, or at least impractical.  If a
single monitor is used, the programmer then has to flip back and forth
between the source code and the interaction loop.  When two monitors
are used, the effect is to waste half a monitor that could otherwise
be used for displaying documentation or something else.

\section{Commenting}

Use a single semicolon to introduce a comment that follows the code on
a line.  Use two semicolons for comments that are not at the top level
in a file and that should be aligned with the code that it comments
on.  Use three semicolons for top-level comments that concern some
top-level forms in a file, but not the entire file.  Use four
semicolons for comments that concern the entire file.

\section{Blank lines}

A single blank line is common in the following situations:

\begin{itemize}
\item Between two top-level forms.
\item Between a file-specific comment and the following top-level
  form.
\item Between a comment for several top-level forms and the first
  of those top-level forms.
\end{itemize}

\section{Spacing}

All modern \commonlisp{} code adheres to the spacing conventions in
this section.

An opening parenthesis is never followed by whitespace, and a closing
parenthesis is never preceded by whitespace.  As a consequence, a
closing parenthesis is never by itself on a line. 

\section{Indentation}
