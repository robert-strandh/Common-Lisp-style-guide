\chapter{Typographical conventions}

\section{Width of a line of code}

Horizontal space is a precious resource that should not be wasted.
The width of a line should preferably not exceed 80 characters.  This
limit used to be hard, because some printers or printer drivers would
truncate longer lines.  Since it is less common to print code these
days, the limit is now soft.  The purpose of keeping lines somewhat
short is so that it is possible on a reasonable monitor to display two
documents side by side.  One document is typically a \commonlisp{}
source file, and the other document is typically the buffer containing
interactions with the \commonlisp{} system.

The systematic use of long lines makes the practice of displaying two
documents side by side impossible, or at least impractical.  If a
single monitor is used, the programmer then has to flip back and forth
between the source code and the interaction loop.  When two monitors
are used, the effect is to waste half a monitor that could otherwise
be used for displaying documentation or something else.

\section{Commenting}

Use a single semicolon to introduce a comment that follows the code on
a line.  Use two semicolons for comments that are not at the top level
in a file and that should be aligned with the code that it comments
on.  Use three semicolons for top-level comments that concern some
top-level forms in a file, but not the entire file.  Use four
semicolons for comments that concern the entire file.

Not only is this practice widely agreed upon, but it is also
recommended by the standard.  See section 2.4.4.2 in the \commonlisp{}
\hs{}.

\section{Blank lines}

A single blank line is common in the following situations:

\begin{itemize}
\item Between two top-level forms.
\item Between a file-specific comment and the following top-level
  form.
\item Between a comment for several top-level forms and the first
  of those top-level forms.
\end{itemize}

It is not common to have blank lines within a single top-level
expression.  If it seems appropriate to divide a top-level function
into sections and to separate these sections by a blank line, it is
usually better to divide the function into several smaller functions
instead.

\section{Spacing}

All modern \commonlisp{} code adheres to the spacing conventions in
this section.

An opening parenthesis is never followed by whitespace, and a closing
parenthesis is never preceded by whitespace.  As a consequence, a
closing parenthesis is never by itself on a line. 

\section{Indentation}

When a form does not fit on a single line, the lines other than the
first one must be \emph{indented} relative to the beginning of the
first one.  The exact indentation depends on the operator of the
form.

If a \emph{function call form} does not fit on a single line, there
are several possible ways of indenting the subsequent lines.  If the
name of the function is very long, but all the arguments will fit on a
line by itself, all the arguments can be put on a second line, like
this:

\begin{verbatim}
(very-long-function-name
 argument-1 argument-2 ... argument-n)
\end{verbatim}

The list of arguments is indented a single position compared to the
first line of the form.

If the argument forms of a function-call form do not all fit on a line
by themselves, but the function name and the longest argument form
will, then the following style is preferable:

\begin{verbatim}
(very-long-function-name argument-1
                         argument-2
                         ...
                         argument-n)
\end{verbatim}

The first argument form is placed after the function name on the first
line and the remaining argument forms are placed each on a subsequent
line, aligned with the first argument form.

When the function takes keyword arguments, each line may contain both
the keyword and the value, like this:

\begin{verbatim}
(very-long-function-name required-argument-1
                         required-argument-2
                         ...
                         required-argument-m
                         :keyword-1 value-1
                         :keyword-2 value-2
                         ...
                         :keyword-n value-n)
\end{verbatim}

For some standard functions that take a single required argument
followed by a number of keyword arguments, in particular
\texttt{make-instance}, the following style can save a significant
amount of horizontal space:

\begin{verbatim}
(make-instance 'class-name
  :initarg-1 value-1
  :initarg-2 value-2
  ...
  :initarg-n value-n)
\end{verbatim}

This style is especially useful when the value forms are themselves
compound forms that occupy a significant amount of horizontal space.
