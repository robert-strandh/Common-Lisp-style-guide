\chapter{Introduction}
\pagenumbering{arabic}%

As many books and articles about programming and software engineering
point out, program code is primarily used as a means of communicating
between software developers, and only incidentally as a means of
giving instructions to a computer.

A consequence of this primary use of software is that the way the code
is written, i.e. its \emph{style}, influences the facility with which
the communication takes place.  Clearly, for code that is being
developed in collaboration with several people, it is desirable that
the style of the code is such that its meaning is clearly and quickly
communicated to people trying to understand it.

A valid concern, then, is to find a style that every developer in a
team can agree upon.  However, since developers come and go, and the
code must be understandable to developers that are not yet part of the
team, their preferences are not known, and can not be taken into
account when a style is agreed upon.

To find a solution to this conundrum, it helps to see the parallels
between code and ordinary written text, and specifically text that is
meant to communicate some technical or scientific subject between an
author and one or more readers.  Authors of such texts simply adhere
to conventions that are \emph{widely agreed upon}.  That way, they can
be fairly certain that potential readers of their works have a very
high probability of understanding the structure of phrases and
sentences.  Though to understand the full meaning of the text, the
reader must obviously also know the subject that is being written
about.

Let us investigate the nature of such widely agreed upon conventions,
and how they become shared to such a high degree.  

Style conventions vary from one language to another.  These variations
can be in typography.  In French typography, for instance, a word and
a following colon are separated by a thin space, but not so in
English.  In German, nouns are capitalized, but not in most other
languages.  But in particular, there are variations in how phrases are
built.  In English you can say ``I was my hands'', whereas in French,
the equivalent phrase (translated into English) would be ``I me wash
the hands'', and in Swedish it would be ``I was the hands [on me]''.

To understand more about style conventions in various languages, it
helps to distinguish between phrases that are \emph{grammatical} and
phrases that are \emph{idiomatic}, and the set of idiomatic phrases is
tiny compared to the set of grammatical phrases.  For some language
such as English, in the first category you will find phrases that
appear as examples in a typical grammar book for that language.
However, whereas most people agree that those phrases are valid, most
of them would never be uttered by anyone.  For a concrete example, in
English, you may say something like ``After a meal, I use my tooth
brush, some tooth paste, and some dental floss so as to avoid having
to see the dentist too often'', and this is an idiomatic phrase.  But
the phrase ``After a meal, I use my dental brush, my dental paste, and
some tooth thread so as to avoid having to see the tooth doctor too
often''.  The latter phrase is grammatical, but it is not idiomatic.
Notice in particular that the choice of words, in this case between
\emph{tooth brush} and \emph{dental brush}, between \emph{dentist} and
\emph{tooth doctor} is completely arbitrary.  Only historical reasons
can explain why one word is acceptable and another word is not.  For
this reason, it is often futile to ask \emph{why} a particular style
is better than some other style.

A situation similar to that of natural languages also holds for
programming languages.  The grammatical phases of a programming
language are those statements and expressions that the compiler must
accept, according to the specification of the language.  The vast
majority of the grammatical phrases of a programming language are not
idiomatic and should not be used in real code.  Just as with natural
languages, the reason a particular phrase is idiomatic is often just
an accident of history.  It is idiomatic not for a particular
intrinsic reason, but just because it is historically what people have
come to agree upon.  The purpose of a style guide, like this one, is
to examine this difference between phrases that are simply
grammatical, and phrases that are also idiomatic.

Style conventions vary over time, and this is normal, both for written
text and for code.  Changes are often very slow, so they are not
always noticed.  However, a comparison between a recent text or
program and one that was written a few decades ago, can be quite
revealing.

For ordinary text, most people are exposed to a vast set of texts in
the form of books and articles.  Therefore, most people are constantly
exposed to works that adhere to the same style conventions.  As a
consequence, these style conventions do not have to be explicitly
pointed out for most people to know them anyway.  Unfortunately,
however, in software development we are faced with teams of developers
with very little experience in reading and understanding code written
by people other than themselves.  There are several reasons for this
undesirable situation.  First, programmers are exposed to code written
by others much later in life than is the case for ordinary text.
Second, courses in programming do not emphasize the necessity of
reading code written by others.  Third, individual software developers
do not routinely read much code written by others.

Because of this lack of exposure to code written by others, we think
that a style guide like this one can be of great help to software
developers.  Ordinarily, a style guide consists mainly of a collection
of rules of what to write and what not to write.  We go one step
further.  When there is a reason for a particular rule, in terms of
how it helps the person reading the code understand it better or
faster, we try to point out this particular reason.  Furthermore, we
try to give references to existing code or to the literature, so that
it is clear that the rules we show are not just our own preferences.
And, when alternative style conventions exist, we point this out,
rather than insisting on one particular alternative.

\section{Conventions in this style guide}

We use \emph{the author} to mean the person who wrote some piece of
code, and we use \emph{the maintainer} to mean the person who tries to
understand the code written by the author, in order to understand that
code, or perhaps even to modify it in various ways.
